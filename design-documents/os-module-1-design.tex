
\chapter{Design of Module \textit{threads}}


\section{Initial Functionality}

Describe briefly what you have been given (have started from) at the beginning of the assignment and the way the existing functionality must be extended.

\section{Data Structures and Functions}

Specify the \textit{data structures} and \textit{functions} involved in your solution and what are they used for. Describe in few words the purpose of new added data structures, fields and functions. 

\begin{lstlisting}
    struct thread
    {

	  ...
      /* the time (in number of ticks) at which a 
	  sleeping thread should wake up */
	  int64_t wakeup_time;		
	  ...

    };
    
    /* a sorted list containing all sleeping 
    (blocked) threads. Sorting is done using 
    thread_wakeup_time_comparison function*/
    static struct list sleep_list;

    /* comparison function to order the sleeping
    threads ascending by their wakeup time */
    list_less_func thread_wakeup_time_comparison;

    /* sets the time at which the thread should 
    wake up and puts the thread in the sleeping 
    threads list */
    void thread_sleep (int64_t wakeup_time);

    /*this function checks if there are sleeping
    threads that should wake up at this moment 
    and calls the function thread_unblock for 
    each of those threads.*/
    void handle_sleeping_threads();    
\end{lstlisting}


\section{Functionality}

Describe \textbf{briefly}, still \textbf{explicitly}, in words and pseudocode the way your solution works. DO NOT INCLUDE detailed code. 

Give examples, if you think they can make your explanation clearer. You are free to use any other techniques (e.g. use-case diagrams, sequence diagrams etc.) that you think can make your explanation clearer. See Figure~\ref{fig:sample-image} below to see the way images are inserted in a Latex file. 

\begin{figure}[h]
	\centering
	\includegraphics[width=0.5\textwidth]{figures/sample-image.pdf}
	\caption{Sample image}
	\label{fig:sample-image}
\end{figure}


Here you have a pseudo-code description of an algorithm taken fom \\ \href{http://en.wikibooks.org/wiki/LaTeX/Algorithms\_and\_Pseudocode\#Typesetting\_using\_the\_program\_package}{http://en.wikibooks.org/wiki/LaTeX}. It uses the \textit{program} package. Alternatively, you can use \textit{algorithmic} or \textit{algorithm2e} packages. 

\begin{program}
\mbox{Example of a pseudo-code algorithm description:}
\BEGIN %
  \FOR i:=1 \TO 10 \STEP 1 \DO
     |expt|(2,i); \\ |newline|() \OD %
\rcomment{This text will be set flush to the right margin}
\WHERE
\PROC |expt|(x,n) \BODY
          z:=1;
          \DO \IF n=0 \THEN \EXIT \FI;
             \DO \IF |odd|(n) \THEN \EXIT \FI;
\COMMENT{This is a comment statement};
                n:=n/2; x:=x*x \OD;
             \{ n>0 \};
             n:=n-1; z:=z*x \OD;
          |print|(z) \ENDPROC
\END
\end{program}


\section{Design Decisions}

Justify your design decisions, specify other design alternatives, their advantages and disadvantages and mention the reasons of your choice.  

\section{Tests}

Describe briefly the tests you are intended to run in order to test the functionality of your implementation.

\section{Observations}

You can use this section to mention other things not mentioned in the other sections. 

You can indicate and evaluate, for instance:
\begin{itemize}
	\item the most difficult parts of your assignment and the reasons you think they were so; 
	
	\item the difficulty level of the assignment and if the allocated time was enough or not; 

	\item particular facts or hints you think we should students to help them solve better the assignment.

\end{itemize}

You can also make suggestions for teacher, relative to the way he can assist more effectively the students.
