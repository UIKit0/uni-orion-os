\chapter{Design of Module \textit{virtualmemory}}


\section{Paging}

At the beginning of this project, it is not possible to load programs that are bigger than the memory size of the system. To be able to do this we want to implement virtual memory in pintos. The most important issue that has to be solved in order to implement virtual memory in pintos is Paging. At this moment, pintos organizes the memory in pages, but it has the following limitations:
\begin{itemize}
 \item the whole executable is loaded at run-time 
 \item the whole executable is unloaded at end of run
\end{itemize}

In order to remove this limitations this part will implement the following three sub parts:
\begin{itemize}
  \item lazy-loading of executables - only a part of executable is loaded at run-time
  \item	eviction - unused parts of executable are unloaded from memory even if the program is running
  \item swap - temporary buffer for pages that were modified since loading and are chosen for eviction
\end{itemize}

\section{Data Structures and Functions}

\textbf{Lazy-loading}

\begin{lstlisting}
 
  //In page.h:
  
  *!!! Copied from VM Laboratory sources !!!*

  // an entry (element) in the supplemental page table
  struct supl_pte {					
	/* the number of the virtual page 
	 (also the hash key) having info about*/
	uint32_t	virt_page_no; 			
	// the virtual address of the page
	void*		virt_page_addr;			
	/* the offset in exe_file where the page starts */
	off_t 		ofs;				
	/* number of bytes to read from the file and
	    store in the page */
	size_t 		page_read_bytes;		
	// number of bytes to zero in the page
	size_t 		page_zero_bytes; 		
	// indicate if the page is writable or not
	bool		writable;		
	// element to insert the structure in a hash list
	struct hash_elem he;				
  };

  /*
    initializes the supplemental page table
  */
  void supl_pt_init(struct hash *);
  
  /* tries to load the given page
     and returns true if succeeded,
     false otherwise */
  bool page_load(uint8_t *upage);


  //In process.h:
  struct process_t {
    /*file used for denying writes 
    and for lazy loading of executable*/
    struct file * exe_file;

  #ifdef VM
    //supplemental page table - used
    //for lazy loading
    struct hash supl_pt;				
  #endif

};
  
\end{lstlisting}

\textbf{Eviction}

\begin{lstlisting}

  //In frame.h:

  struct frame
  {
	//kernel virtual address of page
	void *kpage; 	
	  
	//user virtual address of page
	void *upage;	
	
	//if frame is pinned, the contained page can not be evicted
	bool  pinned;		
	
	struct hash_elem he; //needed for hash_table
  };

  /*allocates a page within a frame
    and returns the allocated frame
    if there is an unused frame.
    Otherwise, it tries to evict a 
    frame and return it. If no frame
    is unused and no frame can be evicted
    it returns NULL. */
  frame* ft_aloc_frame(void *upage);

  /*
   * Frees the memory occupied by
   * frame.
   */
  void 	ft_free_frame(frame *f);  

  /*
   * Pins the frame whose page contains the
   * virtual address given; namely disqualifies 
   * that frame from being evicted by the 
   * LRU algorithm
   */
  void ft_pin_frame(const void *uaddr);
  void ft_unpin_frame(const void *uaddr);


  //initializes the frame-table
  void ft_init();

  //In frame.c:
  /* The frametable */
  static struct hash frame_table;

  /* 
   * Hash function for the frame table.
   * It should be computed using the kpage field of the frame 
   */
  static unsigned 
  frame_hash(const struct hash_elem *f_, void *aux UNUSED);
  
  /* 
   * Frame table utility functions
   */
  static bool 
  frame_less(const struct hash_elem *a_, const struct hash_elem *b_, void *aux UNUSED);
  
  frame* frame_lookup(const void *kpage);

  /* A lock for frame_table synch access */
  struct lock ft_lock;

  /*
   * Finds the least recently used frame
   */
  frame* ft_get_lru_frame();
  
  /* 
   * Tries to evict the frame given. If the 
   * page within the frame was modified, the page
   * is written to swap. 
   */
  bool ft_evict_frame(frame);

\end{lstlisting}

\textbf{Swapping}

\begin{lstlisting}
	
  //In page.h
  
  struct supl_pte
  {
	  ...
	  /* indicates if the page is in swap or on the disk
	      if greater than 0, page is in a swap slot */
	  int	swap_slot;
	  ...
  }
  
  //In swap.h:
  
  /*
    Creates a swap file on the swap
    partition
  */
  swap_init();
  
  /* swaps the page into memory from a
      slot number at kpage address*/
  swap_in(kpage, slot_no);
  
  /* swaps the page out of memory
      into auxiliary storage. Returns
      the number of the slot it occupies
      in swap.
  */
  int swap_out(kpage);
	
\end{lstlisting}

\section{Functionality}

\textbf{Lazy-loading}

\begin{lstlisting}

  //In process.c:
  
  load_segment(...)
  {
    ...
    off_t crt_ofs = ofs;

    file_seek (file, ofs);
    while (read_bytes > 0 || zero_bytes > 0) 
    {
      ...

      // Added by Adrian Colesa - VM

      // Store info about that page to be able 
      // to load it when needed (when a page fault occurs)      
      spte = malloc(sizeof(struct supl_pte));
      spte->virt_page_addr = upage;
      spte->virt_page_no = pg_no(upage);
      spte->ofs = crt_ofs;
      spte->page_read_bytes = page_read_bytes;
      spte->page_zero_bytes = page_zero_bytes;
      spte->writable = writable;
	  spte->swap_slot = -1;
      hash_insert (&thread_current()->supl_pt, &spte->he);

      crt_ofs += page_read_bytes;
      ...
  }


 page_load(upage): returns a boolean value
 {
    p = process_current();

    // Look for the missing page in the supplemental page table
    sup_pte = page_lookup(p, pg_no(upage));
    
    exe_file = p.exe_file;
    page_read_bytes = sup_pte.page_read_bytes;
    page_zero_bytes = sup_pte.page_zero_bytes;
    writable = sup_pte.writable;

    /* Get a page of memory. */
    frame = ft_alloc_frame(PAL_USER, upage);

    if(frame == NULL)
	//no more memory (neither RAM, nor SWAP)
	return false

    uint8_t *kpage = uframe->kpage;
    
    //advance to offset
    file_seek(file, sup_pte.ofs);

    /* Load this page. */
    if( swap_slot < 0 ) {
	file_read(file, kpage, page_read_bytes) != (int) page_read_bytes) 
    }
    else {
	swap_in(kpage, swap_slot);
	sup_pte.swap_slot = -1;
    }
    
    memset(kpage + page_read_bytes, 0, page_zero_bytes);

    /* Add the page to the process's address space. */
    install_page(upage, kpage, writable)

    return true;
}

  //In exception.c:
  
  page_fault()
  {  ...
    if(not_present)
    {
	  process_t *p = process_current();
	  supl_pte *spte = supl_pt_get_spte(p, fault_addr);
	  if(spte == NULL)
	  {
	      INVALID_ACCESS();
	  }
	  else if(!load_page_lazy(p, spte))
	  {
	      //no more memory or kernel bug
	      kill(f);
	  }
    }
    else {
	  //writing r/o page
	  INVALID_ACCESS();
    }
    return;
  }
  
  INVALID_ACCESS()
  {
    if(user)
    {
      thread_exit();
    }
    else
    {
	if(is_kernel_vaddr(fault_addr))
	{
	    kill(f);
	}
	else
	{
	    f->eip = (void *)f->eax;
	    f->eax = -1;
	}
    }
  }
    
\end{lstlisting}

	
\textbf{Eviction}

\begin{lstlisting}

  //In process.c: Change palloc_get_page with ft_alloc_frame 
  //              and palloc_free_page with ft_free_frame  

  ft_alloc_frame(flags, void *upage_addr) : returns a frame*
    void *kpage = palloc_get_page(flags, upage_addr);
    if(kpage != NULL) {
      frame *f = malloc(sizeof frame)
      f.kpage = kpage;
      f.upage = upage;
      f.pinned = false;
      
      lock_aquire(&ft_lock);
      frame.pinned = true;
      insert_hash(f.he);
      frame.pinned = false;	
      lock_release(&ft_lock);
      
      return f;
    }
    else {
      lru_f = ft_get_lru_frame();
      if lru_f != NULL
	  ft_evict_frame(lru_f, false);
	  lru_f.upage = upage;
	  if (flags & PAL_ZERO)
		  memset (lru_f.kpage, 0, 1);
	  frame.pinned = false;
	  return lru_f
    }

    return NULL;
  }
 
  ft_evict_frame(frame)
  {
      if( pagedir_is_dirty(frame.upage) )
      {
	  frame.pinned = true;
	  supl_pte = get_supl_pte(current_process, frame.upage);
	  supl_pte.slot_no = swap_out(frame.kpage)
	  if(supl_pte.slot_no < 0)
		  return false
	  frame.pinned = false;
      }	
      
      pagedir_clear_page(thread_current()->pagedir, f->upage);
      
      return true
  }

  ft_free_frame(frame)
  {
      pagedir_clear_page(thread_current()->pagedir, f->upage);
      palloc_free_page(f->kpage);
      
      //remove frame from frame_table
      frame->pinned = true;
      lock_acquire(&ft_lock);
      hash_delete(&frame_table, &(frame->he));
      lock_release(&ft_lock);
      free(frame);
  }
	
  frame* ft_get_lru_frame()
  {
    while (true) {
      if ( pinned(frame) && 
	    pagedir_is_accessed(thread_current()->pagedir, frame.upage)) 
      {
	pagedir_set_accessed(thread_current()->pagedir, 
			     frame.upage, false);
	continue;
      }     
      else {
	return frame;
      }
    }
  }


  ft_pin_frame(const void *uaddr)
  {
  void *kpage = pagedir_get_page(thread_current()->pagedir, uaddr);
  frame *f = frame_lookup(kpage);
  f->pinned = false;
  }


\end{lstlisting}

\textbf{Swapping}

\begin{lstlisting}
  
  //In swap.c:
  
  // Table used to identify free swap slots
  static struct bitmap swap_table;

  // Block device for storing the swap slots
  struct block *swap_block;

  // Number of pages that fit inside a block sector
  const int PAGES_PER_SECTOR = BLOCK_SECTOR_SIZE / PGSIZE;

  /*
    * Buffer of size BLOCK_SECTOR_SIZE for storing an entire sector.
    * This is used because block_read only works with chunks of BLOCK_SECTOR_SIZE.
    */
  void *bounce_buffer;

  swap_init()
  {
      swap_block = block_get_role(BLOCK_SWAP);
      swap_size = block_size(swap_block) * BLOCK_SECTOR_SIZE;
      bitmap_create( swap_size / PGSIZE );
      bounce_buffer = malloc(BLOCK_SECTOR_SIZE);
  }
  
  void swap_in(kpage, slot_no)
  {
      block_sector_t slot_sector = slot_no / PAGES_PER_SECTOR;
      unsigned slot_offset = slot_no % PAGES_PER_SECTOR;

      // read the sector into the bounce buffer
      block_read(swap_block, slot_sector, bounce_buffer);

      // copy the page from the bounce_buffer to kpage
      memcpy(kpage, bounce_buffer + slot_offset * PGSIZE, PGSIZE);

      // mark the swap slot as empty
      bitmap_reset(swap_table, slot_no);
  }
  
  bool swap_out(kpage)
  {
      // find an empty swap slot
      slot_no = bitmap_scan_and_flip(&swap_table, true);

      if(no slot available) return false;

      block_sector_t slot_sector = slot_no / PAGES_PER_SECTOR;
      unsigned slot_offset = slot_no % PAGES_PER_SECTOR;

      if(no slot available) return -1;

      // read the sector into the bounce buffer
      block_read(swap_block, slot_sector, bounce_buffer);

      // copy the page from the bounce_buffer to kpage
      memcpy(bounce_buffer + slot_offset * PGSIZE,k page, PGSIZE);

      // mark the swap slot as used

      // write the sector back to the swap block

      return true;
  }
	
\end{lstlisting}

\begin{lstlisting}

\end{lstlisting}

\section{Design Decisions}

\textbf{Lazy-loading}

To implement lazy-loading we chose to add a supplemental page table to each process where the information needed to load the page is kept. Keeping only a global supplemental page table is also possible but would make the search in the table a little bit slower, and the manage of the table more complicated.
On the other side, putting the supplemental page table in the thread would duplicate a lot of information in case multithreading user programs are implemented. This is why we chose to put the supplemental page table in the process structure.

The data structure used for the supplemental page table is a hash map, because we need fast access by some page identifier, like the page number. Hashmaps provide this advantage for less space than simple arrays.

\textbf{Eviction}

We chose to implement the frame table as a hash map. Other options would have been a sparse array or a list. The problem with the array is that it would occuppy the same space whether the frame table is full or not. However the benefit of the array is that it has constant access by index. The list, on the other side occuppies less space than the vector in the when the memory is not occupied, but does not provide constant access time. Therefore, we chose to implement the frame table as a hash map.
Altough normally a frame table maps a physical address to a virtual address we chose the key of the frame table to be a virtual kernel address, because of the strong correlation in pintos between the physical address and the kernel virtual address.

We will be using the LRU approximation algorithm called second-chance. This works by iterating over all the frames present in the main memory and checking whether they have been accessed recently or not. The first frame not recently accessed is chosen to be swaped out. Each time we pass over a recently accessed frame, we change its accessed status to false. The iteration works in a circle loop, so in the worst case, the starting frame will be chosen to be swaped out (having it's accessed status set to false the first iteration).

This is implemented in the ft\_get\_lru\_frame function, which uses pagedir\_set\_accessed and pagedir\_is\_accessed functions to determine the accessed status of a page.

\textbf{Swapping}
The swap table is implemented using a bitmap, because we only need to keep track whether a slot is occupied or not, therefore, a bit/slot is enough. The advantage of this solution against to other solutions is that it occupies less space.

We will be using the SWAP block device to store the swaped pages. Unfortunately, the block drivers only work by reading / writing whole sectors from / to main memory. This means we need to keep an additional bounce buffer in which to store the entire sector containing the page. This requires an additional memcpy to insert / extract the page from the bounce buffer. This is unfortunate, but we found no better solution, since it's also the way the filesystem is implemented.

\section{Tests}

After the implementation of paging, all tests from project should still pass plus some tests regarding paralellism.

page-linear.c - Encrypts, then decrypts, 2 MB of memory and verifies that the values are as they should be.
page-shuffle.c - Shuffles a 128 kB data buffer 10 times, printing the checksum after each time.
page-parallel.c - Runs 4 child-linear processes at once.
page-merge-seq.c - Generates about 1 MB of random data that is then divided into 16 chunks.  A separate subprocess sorts each chunk in sequence.  Then we merge the chunks and verify that the result is what it should be.
page-merge-par.c - The same in parallel.

\section{Stack Growth}
\subsection{Initial Functionality}

The stack is just one page and it can't grow any higher. The stack is at the top of the user
virtual address space.

\subsection{Data Structures and Functions}

We need to allocate more pages for the stack when the stack goes higher. Also we need to impose some absolute limit so it can not get higher. So we need to detect when a page fault occurred because of the growth of the stack. 

\begin{lstlisting} 

	//thread.h

	//add a new field esp that will holds the stack pointer
	//when an exception causes a switch from user mode to
	//kernel mode
	struct thread {
	  ...
	  void* esp;
	  void* last_stack_page;
	  int numberOfStackGrows;
	};

	//process.c

	//Grows the stack. Esp is the stack pointer and size
	//is the size of offset in the new page.
	//Returns true if the stack can grow, false otherwise
	bool grow_stack( void **esp, size_t size );

	//exception.c

	//Detects if stack generates the page fault. Fault_addr is the 
	//that was accessed to cause the page fault.
	//Returns true if the page fault was generated by the
	//grow of the stack, false otherwise.
	bool is_stack_page_fault( struct intr_frame *f, 
					 void* fault_addr ); 
	
\end{lstlisting}


\subsection{Functionality}
 
\textbf{In syscal\_handler:}
	  \\1. Save the stack pointer in the struct thread

\begin{lstlisting}
	static void syscall_handler( struct intr_frame *f )
	{
	  current_thread()->esp = f->esp;
	}
\end{lstlisting}


\textbf{In page\_fault:}
	  \\1. Get the address that generate the page fault.
	  \\2. Verify if stack cause the page fault by comparing the esp from current with the page fault address. If the difference between them is 32 or 4 then the page fault was cause by the grow of the stack.
	  \\3. If true than the call the method stack\_grow.
	  \\4. Verify the size of the stack, if the size if higher than 8MB then destroy process.
	  \\5. If false continue to determin the reason for the page faul.

\begin{lstlisting}

	bool is_stack_page_fault( void* fault_addr )
	{
	  void* esp = current_thread()->esp;
	  if ( esp - fault_addr == 32  || esp - fault_addr == 4 )
	  {
	    if ( stack_grow() )
	    {
	      return false;
	    }
	    return true;
	  }
	  return false;
	}

\end{lstlisting}


\textbf{In file process.c: }

We will create a function that will grow the stack. The function will be something like this:
      \\1. Get a free page. 
      \\2. If the page is not null, try to connect the  upage with the kpage.
      \\3. If is not possible to connect them dealloc the frame with the page and return false.
      \\4. Returns true is everthing works.

\begin{lstlisting}
    bool stack_grow()
    {
      if ( thread_current()->numberOfStackGrows > STACK_MAX_NUMBER_OF_PAGES )
      {
	  return false;
      }

      void *upage = thread_current()->last_stack_page;
      frame *frame = ft_alloc_free( true, upage );
      uint8_t *kpage = frame->kpage;
 
      if ( kpage != NULL )
      {
	if ( install_page ( ( ( uint8_t* ) upage  ) - PGSIZE, kpage, true )
	{
	  thread_current()->last_stack_page = thread_current()->last_stack_page - PGSIZE;
	  thread_current()->numberOfStackGrows++;
	  return true;
	}
	else
	{
	  ft_free_frame( frame );
	  return false;
	}
      }
      return false;
    }

\end{lstlisting}


\subsection{Design Decisions}

We had to modified the method page\_fault because here we can see the cause of the page fault. Also in this method are tested other possible causes of the page fault. And add a new method in the file process.c named stack\_grow because in this file the stack is initialize and this method depends on the process.


\subsection{Tests}

The following tests are: 

\textbf{pt-grow-stk-obj} - Allocates and writes to 64 kB objects on the stack. 

\textbf{pt-grow-bad} - Reads from an address 4,096 bytes below stack pointer. 

\textbf{pt-grow-pusha} - Expand the stack by 32 bytes all at once using PUSHA instruction. 

\textbf{pt-grow-stack} - Demonstrate that the stack can grow. 

\textbf{pt-grow-stk-sc} - This test checks that the stack is properly extended even if the first access to a stack location occurs inside a system call. 



\section{Memory Mapped Files}
\subsection{Initial Functionality}

We have open/read/write file syscalls. We know for each process its table of opened files.


\subsection{Data Structures and Functions}

We need the following data structures and functions for managing memory mapped files: 

\begin{lstlisting}
	//process.h
	struct process_t {
		list mmap_list;
	};

	typedef int mapid_t;
	struct mapped_file {
		mapid_t id;
		int fd;
		void *user_provided_location;
		size_t file_size;
		enum mapped_file_status status; //optional
		struct list_elem lst;
	};

	//syscall.c
	static void syscall_mmap(struct intr_frame *f);
	static void syscall_munmap(struct intr_frame *f);
\end{lstlisting}
	


\subsection{Functionality}

\subsubsection{ syscall\_mmap }
\textbf{In syscall\_mmap:}
	\\1. Check if the fd is valid.
	\\2. Inform the pages starting from the user provided address that they valid, and not present. We should also check if they aren't used already.
	\\3. If there is no error, make a mapped\_file entry and add it to the current process list.
	\\4. Return the id.
	\\5. We should add/increment a reference to the fd table such that the file is not truly closed until nothing references it anymore.



	\textbf{Somewhere in swap:}
	\\1. We check if the evicted page provided belongs to a memory mapped file. If yes we write that page on the harddisk. Careful with page/file bounds.
	\\2. If the requested page belongs to a memory mapped file, we just read it from the harddisk. Fill with 0s the padding if necessary.

The memory mapped file page checking could be done in the following way: We know the current process in swap/evict because it's inside an exception handler, thus we know how to get to the mapped\_file's list. We simply make range search on each mapped\_file element, since we know that the files are mapped in contiguous regions.

As follows:
\begin{lstlisting}	
	/* may return null when the mapped file is not present  */
	mapped_file *GetMappedFileFromPagePointer(void *pagePointer) {
		process_t *cp = process_current();
		
		foreach (mmentry : cp->mmap_list) {
			if(mmentry->pagePointer < 
				user_provided_location 
				 && user_provided_location < 
					mmentry->pagePointer + 
						mmentry->file_size) {
				return mmentry;
			}
		}
		return NULL;
	}
\end{lstlisting}
	3. The evict/add\_page\_from\_hdd could find out, by using this function what to do next.
\begin{lstlisting}
	ft_evict_frame()
		if( pagedir_is_dirty(pagePointer) || GetMappedFileFromPagePointer(pagePointer))
		{
			//swap ...
		}
\end{lstlisting}

\subsubsection{ syscall\_munmap }
\textbf{In syscall\_munmap:}
	\\1. Write what's in the memory to the file. 
	\\2. Remove the mmentry from the list.

	 \textbf{In process\_exit:}
	\\1. Call munmap for each entry in the process mmap\_list;


\subsection{Design Decisions}
We keep the mapped file list in each process because it's easier to manage the lifetime of the list entries. It's also a performance gain because if we would use a global list, searching could become very slow for a process that doesn't own any mapped files. That searching is done inside an exception handler! An alternative to this would be to use the supplemental page table to hold the necessary data.
\\When unmapping we could forcibly evict all the pages that belong to the file instead of in place writing.
\subsection{Tests}
All the mmap from tests/vm.


\textbf{mmap-bad-fd} - Tries to mmap an invalid fd, which must either fail silently or terminate the process with exit code -1.


\textbf{mmap-clean} - Verifies that mmap'd regions are only written back on munmap if the data was actually modified in memory.


\textbf{mmap-close} - Verifies that memory mappings persist after file close.


\textbf{mmap-exit} - Executes child-mm-wrt and verifies that the writes that should  have occurred really did.


\textbf{mmap-inherit} - Maps a file into memory and runs child-inherit to verify that  mappings are not inherited.


\textbf{mmap-misalign} - Verifies that misaligned memory mappings are disallowed.


\textbf{mmap-null} - Verifies that memory mappings at address 0 are disallowed.


\textbf{mmap-over-code} - Verifies that mapping over the code segment is disallowed.


\textbf{mmap-over-data} - Verifies that mapping over the data segment is disallowed.


\textbf{mmap-overlap} - Verifies that overlapping memory mappings are disallowed.


\textbf{mmap-over-stk} - Verifies that mapping over the stack segment is disallowed.


\textbf{mmap-read} - Uses a memory mapping to read a file.


\textbf{mmap-remove} - Deletes and closes file that is mapped into memory and verifies that it can still be read through the mapping.


\textbf{mmap-shuffle} - Creates a 128 kB file and repeatedly shuffles data in it through a memory mapping.


\textbf{mmap-twice} - Maps the same file into memory twice and verifies that the same data is readable in both.


\textbf{mmap-unmap} - Maps and unmaps a file and verifies that the mapped region is inaccessible afterward.


\textbf{mmap-write} - Writes to a file through a mapping, and unmaps the file, then reads the data in the file back using the read system call to verify.


\textbf{mmap-zero} - Tries to map a zero-length file, which may or may not work but  should not terminate the process or crash.  Then dereferences the address that we tried to map, and the process must be terminated with -1 exit code. 

\section{Accessing user memory}

Accessing user memory during a syscall was already implemented in project 2: userprog. Synchronization related issues are solved by paging design.








	

