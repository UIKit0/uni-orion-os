\chapter{Design of Module \textit{File System}}


\section{Indexed and Extensible Files}
\subsection{Initial Functionality}
  
Currently the file are allocated like continous sectors. If there aren't n contiunous blocks free than a 
file which has the size n * BLOCK\_SECTOR\_SIZE can not be allocated. We have to change this implementation
so a file can be allocated in n noncontiunous blocks.

\subsection{Data Structures and Functions}

We have to modify the structure:

\begin{lstlisting}

  struct inode_disk
  {
    ...
    block_sector_t next_sector; /* we store the address of the next sector */
    uint32_t unused[124]	      /* we modify the size of the array so that the size of inode_disk is still 
				BLOCK_SECTOR_SIZE */
  }

\end{lstlisting}

We have to modify the functions:
\begin{lstlisting}
 
  static block_sector_t
  byte_to_sector( const struct inode *inode, off_t pos )
  {
    ...
    if ( pos < get_size( inode->data )
    {
	get_sector( inode->data, pos / BLOCK_SECTOR_SIZE ); 
    }
    ...
  }

  bool
  inode_create( block_sector_t sector, off_t length )
  {
    ...
    for ( i = 0; i < sectors; ++i )
      block_write( fs_device, get_sector( disk_inode->start, i ), zeros );
    ...
  }

  off_t
  inode_length( const struct inode *inode )
  {
    return get_size( &inode->data );
  }

  void
  inode_close( struct inode *inode )
  {
    ...
    free_map_release( inode->sector, 1 );
    int number_of_sectors = get_number_of_sectors( inode->data );
    block_sector_t sector = inode->data.start;
    int i;
    for ( i = 0; i <= number_of_sectors; i++ )
    {
      free_map_release( sector, 1 );
      sector = sector.next_sector;
    }
    ...
  }

\end{lstlisting}


We have to create the following methods:
\begin{lstlisting}

  /* Gets the inode_disk, and a index representing the index of the sector which 
  inode stores.
     Returns the index sector */
  static block_sector_t
  get_sector( struct inode_disk* inode_disk, int n )
  {
    if ( n == 0 )
    {
      return inode_disk->start;
    }
    else
    {
      get_sector( &inode_disk->next_sector, n - 1 );
    }
  }

  /* Gets the size of the file */
  off_t
  get_size( struct inode_disk *disk_inode )
  {
    off_t size = 0;
    
    if ( disk_inode->next_sector == 0 ) /* Is the last sector */
    {
      size = disk_inode->length;
    }
    else
    {
      size = get_size( disk_inode->next_sector ) + disk_inode->length;
    }
    return size;
  }

  /* Returns the number of sectors that inode has; */
  off_t
  get_number_of_sectors( struct inode_disk *disk_inode )
  {
    return get_size( disk_inode ) / BLOCK_SECTOR_SIZE;
  }

  
  bool
  extend_size( struct inode_disk *disk_inode )
  {
  }

\end{lstlisting}

  

\subsection{Functionality}

Functionality
  

\subsection{Design Decisions}

Design Decisions
  

\subsection{Tests}

Tests

\section{Subdirectories}
\subsection{Initial Functionality}

Initial Functionality
  

\subsection{Data Structures and Functions}
  
Data Structures and Functions

\subsection{Functionality}
  
Functionality

\subsection{Design Decisions}
  
Design Decisions

\subsection{Tests}

Tests

\section{Buffer Cache}
\subsection{Initial Functionality}

Initial Functionality
  

\subsection{Data Structures and Functions}
  
Data Structures and Functions

\subsection{Functionality}
  
Functionality

\subsection{Design Decisions}
  
Design Decisions

\subsection{Tests}

Push to a new Page

\section{Synchronization}
\subsection{Initial Functionality}

Initial Functionality
  

\subsection{Data Structures and Functions}
  
Data Structures and Functions

\subsection{Functionality}
  
Functionality

\subsection{Design Decisions}
  
Design Decisions

\subsection{Tests}

Tests