\chapter{Design of Module \textit{File System}}


\section{Indexed and Extensible Files}
\subsection{Initial Functionality}
  
Currently the file are allocated like continous sectors. If there aren't n contiunous blocks free than a 
file which has the size n * BLOCK\_SECTOR\_SIZE can not be allocated. We have to change this implementation
so a file can be allocated in n noncontiunous blocks.

\subsection{Data Structures and Functions}

We have to modify the structure:

\begin{lstlisting}

  struct inode_disk
  {
    ...
    block_sector_t next_sector; /* we store the address of the next sector */
    uint32_t unused[124]	/* we modify the size of the array so that the size of inode_disk is still 
				BLOCK_SECTOR_SIZE */
  }

\end{lstlisting}

We have to modify the functions:
\begin{lstlisting}
 
  static block_sector_t
  byte_to_sector( const struct inode *inode, off_t pos )
  {
    ...
    if ( pos < get_size( inode->data )
    {
	get_sector( inode->data, pos / BLOCK_SECTOR_SIZE ); 
    }
    ...
  }

  bool
  inode_create( block_sector_t sector, off_t length )
  {
    if ( free_map_try_allocate( sectors, &disk_inode ) )
    ...
    for ( i = 0; i < sectors; ++i )
      block_write( fs_device, get_sector( disk_inode->start, i ), zeros );
    ...
  }

  off_t
  inode_length( const struct inode *inode )
  {
    return get_size( &inode->data );
  }

  void
  inode_close( struct inode *inode )
  {
    ...
    free_map_release( inode->sector, 1 );
    int number_of_sectors = get_number_of_sectors( inode->data );
    block_sector_t sector = inode->data.start;
    struct inode_disk disk_inode;
    int i;
    for ( i = 0; i <= number_of_sectors; i++ )
    {
      block_read( fs_device, sector, &disk_inode );
      free_map_release( sector, 1 );
      sector = disk_inode.next_sector;
    }
    ...
  }

  off_t
  inode_write_at( struct inode* inode, const void* buffer, off_t size,
		  off_t offset )
  {
    /* Get current lenght of the file
    Calculate the difference between current length and offset + size
    If less then 0, than add new sectors to the files
    Fill the sectors with 0
    Write data from buffer at offset 
    Update the lenght of the las inode_disk*/
  }

  off_t
  inode_read_at( struct inode* inode, void* buffer, off_t size,
		 off_t offset )
  {
    /* Get current length of the file
      Find if the offset is greater and current length
      If true then return 0
      Otherwise add data in buffer until reach EOF
      or until written bytes equals with size.
    */
  }

\end{lstlisting}


We have to create the following methods:
\begin{lstlisting}

  /* Gets the inode_disk, and a index representing the index of the sector which 
  inode stores.
     Returns the index sector */
  static block_sector_t
  get_sector( struct inode_disk* inode_disk, int n )
  {
    if ( n == 0 )
    {
      return inode_disk->start;
    }
    else
    {
      struct disk_inode disk_inode;
      block_read( fs_device, inode_disk->next_sector, &disk_inode );
      get_sector( &disk_inode, n - 1 );
    }
  }

  /* Gets the size of the file */
  off_t
  get_size( struct inode_disk *disk_inode )
  {
    off_t size = 0;
    
    if ( disk_inode->next_sector == 0 ) /* Is the last sector */
    {
      size = disk_inode->length;
    }
    else
    {
      struct disk_inode read_disk_inode;
      block_read( fs_device, disk_inode->next_sector, &read_disk_inode );
      size = get_size( &read_disk_inode ) + disk_inode->length;
    }
    return size;
  }

  /* Gets the last sector */
  block_sector_t
  get_last_sector( struct inode_disk* disk_inode )
  {
      return get_sector( disk_inode, get_number_of_sectors( disk_inode ) );
  }

  /* Gets the last inode_disk */
  inode_disk*
  get_last_disk_inode( struct inode_disk* disk_inode )
  {
    while( disk_inode->next_sector != 0 )
    {
	struct inode_disk read_disk_inode;
	block_read( fs_device, disk_inode->next_sector, &read_disk_inode );
	return get_last_disk_inode( read_disk_inode );
    }
    return disk_inode;
  }

  /* Returns the number of sectors that inode has; */
  off_t
  get_number_of_sectors( struct inode_disk *disk_inode )
  {
    return get_size( disk_inode ) / BLOCK_SECTOR_SIZE;
  }

  /* Adds cnt sectors to the disk_inode */
  bool
  extend_size( struct inode_disk *disk_inode, size_t cnt )
  {
    return free_map_try_allocate( cnt, get_last_disk_inode( disk_inode );
  }

  /* Tries to allocate cnt sectors */
  bool
  free_map_try_allocate( size_t cnt, inode_disk *disk_inode )
  {
    block_sector_t sector;
    size_t total_number_of_sectors_allocated = 0;
    size_t blocks_to_allocate = 1;
    bool first_time = true;
    while ( total_number_of_sectors_allocated != cnt )
    {
	while ( !free_map_allocate( blocks_to_allocate, &sector ) || blocks_to_allocate == 0 )
	{
	    --blocks_to_allocate;
	}
	if ( blocks_to_allocate == 0 )
	{
	  return false;
	}

	if ( first_time )
	{
	  disk_inode->start = sector;
	  disk_inode->next_sector = 0;
	  first_time = false;
	}
	else
	{
	  get_last_disk_inode( disk_inode ).next_sector = sector;
	  
	}

	total_number_of_sectors_allocated += blocks_to_allocate;
	//blocks_to_allocate = cnt - total_number_of_sectors_allocated;
    }
  }

\end{lstlisting}

  

\subsection{Functionality}

When it tries to seak after the end of file, the method inode\_read\_at shouldn't break and let the process to read
at that position. This type of operation should not extend the size of the file. But when it tries to write to a postion
that is pas EOF, than the file is extended to that position plus the number of bytes written. We extended the file with
( offset + bytesWritte - file\_length ) / BLOCK\_SECTOR\_SIZE sectors. These blocks contains only zeros. And also we have to
update the length of the file.
  

\subsection{Design Decisions}

We don't store any information regarding the type of the sector in the inode\_disk. We store this information in dir\_entry.

\subsection{Tests}

Tests



\section{Subdirectories}

\subsection{Initial Functionality}

Currently, the root directory is represented by an inode stored at segment 1 which contains enough information to store 16 dir\_entries. Each dir\_entry represents a file. This means the filesystem does not support subdirectories and only supports 16 files inside the root directory. Moreover, the file names are restricted to 14 characters.
  
\subsection{Data Structures and Functions}
  
\begin{lstlisting}
// ---------------------------------------------
// in process.h
// ---------------------------------------------

struct process_t {
...
#ifdef FILESYS
    // current directory
    struct dir *current_directory;
#endif
};

struct fd_list_link {
    struct list_elem l_elem;
    int fd;
    bool is_file;
    union {
	struct dir *dir;
	struct file *file;
    }
    bool mapped;
};

// ---------------------------------------------
// in directory.c
// ---------------------------------------------

struct dir 
{
    struct inode *inode;
    off_t pos;
    struct dir *parent;
};

struct dir_entry 
{
    block_sector_t inode_sector;
    char name[NAME_MAX + 1];
    bool is_file;
    bool in_use;
};

// ---------------------------------------------
// in directory.h
// ---------------------------------------------

/**
 *  Used to identify the type of a path.
 */
enum path_type {
  T_FILE,
  T_DIR,
  T_INVALID
};

/**
 *  Identifies the type of a path.
 */
path_type identify_path_type(const char *path);

/**
 *  Interprets path and attempts to create a
 *  dir structure representing it.
 *  Returns NULL if the path is invalid
 */
dir *parse_dir_path(const char *path);

/**
 *  Interprets path and attempts to create a
 *  file structure representing it.
 *  Returns NULL if the path is invalid
 */
file *parse_file_path(const char *path);

/**
 *  Adds a new subdirectory. Returns false if the name 
 *  is already used for another entry.
 */
bool dir_add_subdir(dir *dir, const char *name);

/**
 *  Attempts to remove a subdirectory. Returns false
 *  if the removal failed. This can happen if the subdirectory
 *  is not empty.
bool dir_remove_subdir(dir *dir, const char *name);

// ---------------------------------------------
// in userprog/syscall.h
// ---------------------------------------------

static void syscall_chdir(struct intr_frame *f);
static void syscall_mkdir(struct intr_frame *f);
static void syscall_readdir(struct intr_frame *f);
static void syscall_isdir(struct intr_frame *f);
static void syscall_inumber(struct intr_frame *f);

\end{lstlisting}


\subsection{Functionality}

For subdirectories, we chose to leave the inode\_disk structure intact, but update the dir\_entry instead. This structure is saved on disk, and represents an entry in a directory. The entry can either be a file or a directory so we must make this distinction. We chose to add an enum to the entry structure which indicates whether it's a file or directory. This allows us to treat the contents represented by the inode of the dir\_entry as either a file or a list of other dir\_entries.

The path parsing functions take as input a char* containing a path, handle the ``.'' and ``..'' special names, and attempt to walk the path by starting at the root path inode. If a name in the path is not found, or a file is found where a directory is expected, the functions return NULL.

When a new process is created, we use the path parsing functions to create a dir* from the executable path. This dir* is stored as the process's current directory. It's also used by the path parsing functions when the special name ``.'' is required.

The file descriptor has been updated to be aware of whether it represents a file or a directory. Old syscalls are also updated to use the path parsing functions and produce a corresponding file descriptor.

The remove filesys call will make use of the dir\_remove\_subdir to remove empty directories.

\subsection{Design Decisions}
  
We chose to make the distinction between a file and a folder in the dir\_entry structure as opposed to the disk\_inode structure, because it keeps the meaning of each structure. Unfortunately, this means that an inode does not contain information about whether it's contents represent a file or a directory. We believe this to be acceptable, since it's the operating system's responsibility to interpret the directory structure, and it should not rely only on inode information. 

Storing the is\_file flag in both the disk\_inode and dir\_entry could lead to a desynchronization between the two, but would allow the operating system to interpret the information represented by the inode. But since this information is used in a context dependent fashion, the type of content represented by the inode should already be known, so storing this information inside the inode would most likely be useless.


\section{Buffer Cache}
\subsection{Initial Functionality}
Implement a caching mechanism for blocks. Read and write operations would go through this cache. There must be an eviction mechanism if the cache is full.
There will be a thread that will write periodically the cache on the physical device.
  

\subsection{Data Structures and Functions}
\begin{lstlisting}
int const block_size_in_sectors = 4;
int const cache_size_in_sectors = 64;
int const sector_size = 512;
int const cache_size_in_blocks = cache_size_in_sectors /  block_size_in_sectors;

struct block {
	char data[sector_size * block_size_in_sectors]; 
};

struct block_supl {
	bool present;
	bool accessed;
	bool dirty;
	bool pinned; //or a mutex true while there is a block operation
	int sector_start;
	int block_start;
};

struct buffer_cache {
	struct	block cache[cache_size_in_blocks]; 
	struct block_supl cache_aux[cache_size_in_blocks]; // or a list
};

//public interface
void write_block(int  block_address, void *buffer);
void read_block(int  block_address, void *buffer);
void dump_cache(); //dumps all the blocks on the hdd

//internal
int get_block_cached_index(int block_address); //return the  
  //block_index_in_cache of the block, 
  //if the block is not in cache returns -1
int evict(int  block_index_in_cache); //evicts a block and returns the  
  //block_index_in_cache of the freed block.. It may writeback data.
int lru(); // returns a single index used by evict...
int read_ahead_asynch(int block_adress); //will do a read ahead asynchronously. 
  //This will append a request to a list of request in the cache thread 
  //(the one which is responsible with write periodically)

void dump_block(int block_index_in_cache);
\end{lstlisting}


\subsection{Functionality}
All read/write operations except swapping, mmap, (virtual memory ones) will go through here.  


\paragraph {write\_block}
\begin{enumerate}
	\item check if the block is in the cache... if yes override the data
	\item if not in the cache... evict one block and write the data in the block\_slot
\end{enumerate}

\paragraph {read\_block}
\begin{enumerate}
	\item check if the block is in the cache, if yes, return it
	\item if not evict, and read from PH device
\end{enumerate}

\paragraph {dump\_cache}
\begin{enumerate}
	\item for each block in the cache
	\item dump\_block(block)
\end{enumerate}


\paragraph {get\_block\_cached\_index}
\begin{enumerate}
	\item will search in the cache\_aux for the corresponding index in the cache.
\end{enumerate}

\paragraph {evict}
\begin{enumerate}
	\item lru returns a (a block\_index)
	\item dump\_block(block)
\end{enumerate}

\paragraph {lru}
\begin{enumerate}
	\item determines the next block to be evicted, it can be an empty block, so dump block will not write anything
\end{enumerate}

\paragraph {dump\_block:}
\begin{enumerate}
	\item if the block has "present" and dirty flag set will write it on the hdd.
\end{enumerate}

All the global data structure writes / reads all synchronized with lock. It could be global or more granular (in a cache block).

There will be a thread with alarm clock which will dump all the blocks periodically.  (the buffer cache thread)
There will also be a dump in  filesys\_done() call.




\subsection{Design Decisions}

We tried to maximize the transparency of the caching, so that the read/write functions user doesn't know anything about it. By synchronizing the global data access we enforce the multithreaded safety.


\subsection{Tests}


\section{Synchronization}
\subsection{Initial Functionality}

At the beginning of this project, our filesystem needs external synchronization to function properly. Actually at this moment only on process can be in filesys code at a time whether this condition is needed or not. We want to implement a finer-grained synchronization, that is internal (a. k. a. the kernel doesn't have to worry about synchronizing access to files).
  

\subsection{Data Structures and Functions}
  
\begin{lstlisting}
  // in memory inode 
  struct inode 
  {
    ...
    /* a lock / inode used for directory/files
       operations synchronizations. This is 
       stored in in-memory inode. Should not be
       saved on the disk. */
    struct lock inode_lock; 
  }

  //in cache
  struct cache_block
  {
    ... 
    // a lock / block 
    struct lock block_lock;
  }

\end{lstlisting}

\subsection{Functionality}
  
\begin{lstlisting}
  // In directory.c
  dir_operation
      lock_acquire(dir->inode.inode_lock);
      ...
      lock_release(dir->inode.inode_lock);
  dir_operation_end

  inode_write_at()
    if( offset < size )
      lock_acquire(inode.inode_lock);
      extend_file() // do not update file's length yet
      block_write
      inode.length = new_length;
      lock_release(inode.inode_lock);
    
  //In cache.c:
  cache_operation
      lock_acquire(dir->inode.inode_lock);
      ...
      lock_release(dir->inode.inode_lock);
  cache_operation_end

\end{lstlisting}

\subsection{Design Decisions}
  
We chosed to keep an lock per inode because the simplicity of the solution. This also respects the requirements that operations on different entities should be done in parallel, while operations on the same entity may wait one after another. The lock has to be kept in the inode and not in the file or dir structure because the file and dir structure are allocated for each process separately, even if they might point to the same inode.
Synchronization code is put in directories and not in the inodes, because, there are operations on inodes which should be able to happen concurently like reading/writing from/to a file.

We used the same rationale for caching synchronization.

\subsection{Tests}

syn-read - Creates 10 children and calls file\_read in each of them.The contents should match.
syn-write - Creates 10 children and calls file\_write in each of them.Then compares the results.
syn-remove - Verifies that a file that has been deleted is still able to be readen or written to.